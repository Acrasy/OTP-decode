\documentclass{scrartcl}

\usepackage[utf8]{inputenc}
\usepackage[T1]{fontenc}
\usepackage[ngerman, english]{babel}
\selectlanguage{english}
\usepackage{listings}
\usepackage{hyperref}
\usepackage{upquote}

\lstset{ %
	frame=single,
	numbers=left,
	breaklines=true,
	breakautoindent=false,
	breakatwhitespace=true,
	% keepspaces=true,
	basicstyle=\ttfamily,
	upquote=true,
}

\newcommand\solution[2]{{\paragraph{#1}#2}}
\newcommand\todo[1]{TODO: #1}


\title{Assignment 2 - One-Time-Pad}
\author{$<$Firstname$>$ $<$Lastname$>$, $<$Student ID$>$}
\date\today{}

\begin{document}

\maketitle

\begin{abstract}
You are visiting Bikini Bottom and staying with your friend Spongebob.
Nowadays, Spongebob and Patrick are secret agents and practice excessively
to encrypt and decrypt messages. For you it is deadly dull in Bikini
Bottom, because they do it all day. Since you have some background in
crypto, you become interested in what they use for encryption. Without
Spongebob being aware of it (which wasn't so difficult because it is
Spongebob...), you copy 10 encrypted messages. You could also cast a glance
at some code that was only partly hidden at the desk.

You immediately get the suspicion that SpongeBob and Patrick could be using the
``unbreakable'' One-Time-Pad. But, you suspect a serious flaw in their execution and
decide to break the next secret message (target ciphertext) and confront
your friends with the result.

Please document your findings and your solution by filling out this template
and upload the resulting PDF to TUWEL.
\end{abstract}


\section*{Overview}
Note: The paragraphs marked as \emph{TODO} are instructions and provide guidance to what
information should be included in the final report. You should remove them before submitting.

\todo{Describe your general approach to solving this assignment. Include any used 
tools, websites or guides that helped you. You can use \lstinline{\\footnote\{reference\}} to 
include references to resources that you used.\footnote{for example: \url{http://example.org}}}

\todo{In this assignment you most certainly will have to write some code to
break the encryption. Please include it using a listing as shown below:}

\begin{lstlisting}
if (corp == "evil") {
    hack_backups()
}
\end{lstlisting}


\section{One-Time-Pad}

\section*{Assignment}
Download the ten ciphertexts and one target ciphertext from the website
linked to in TUWEL. Your goal is to break the encryption and obtain the
original plaintexts. To show you broke the encryption it suffices to provide
the decrypted of the target ciphertext (11th message), but feel free to include all of
course.

The code you spotted on SpongeBobs desk is following line inside the encryption
routine:
\begin{lstlisting}
sprintf("%02X", ord($msg{$i}) ^ ord($key{$i}));
\end{lstlisting}

Hints:
\begin{itemize}
    \item The plaintexts are in English.
    \item You might want to write some lines of code to decrypt the secret message.
    \item Don't be intimidated by the length of the texts! You know, the longer the text, the easier it is to decrypt. Again, you may find it easier if you write yourself a little program.
\end{itemize}

\section*{One-Time-Pad Encryption}
\todo{Describe how encryption using one-time-pads works. What, besides the
plaintext, is needed to encrypt a message? What properties must it have in
relation to the plaintext?}\\
\todo{Name the main benefit of one-time-pad encryption.}\\
\todo{Name at least two drawbacks when using one-time-pad to encrypt messages.}

\section*{Breaking One-Time-Pad}
\todo{Why are you sure that you can break the ``unbreakable'' message? It might
be something that Spongebob and Patrick forgot or could have done wrong.}\\
\todo{Elaborate a bit on the mistake they made. What consequence does the
mistake result in. How can you exploit it?}\\
\todo{Describe in detail how you decrypted the ciphertexts.}

\section*{Results}
\solution{Key}{Insert the recovered key here}
\solution{Source}{News article from which the messages originate.}
\solution{Target plaintext}{Insert the plaintext from the 11th ciphertext here}

\todo{The texts appear to originate from a news article. Find the news article
of the target ciphertext and read it (don't forget to include the link to the
article and title in the report).}\\
\todo{If you wrote some code to break the encryption please include it here as
a listing. You can choose any language you want to.}
\end{document}
